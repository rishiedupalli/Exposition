\documentclass[12pt,a4paper,english,oneside]{book}
\usepackage[titletoc]{appendix}
\usepackage{amsmath}
\usepackage{amssymb}
\setlength{\parindent}{0cm}
\setlength{\parskip}{0,5cm}

\begin{document}

\title{Competition Mathematics}

\author{Rishi C. Edupalli}
\date{}

\maketitle

\tableofcontents

\chapter*{Preface}

Calling this a book would be a grave injustice to books. It would be best defined as an encyclopedia or, better yet, as lecture notes. It is not recommended that one use this for learning competition math from the beginning and should be used primarily as a reference. As the primary use of whatever one had decided to call this is as a reference, we include little worked examples and offer practice problems with no solutions in an appendix sorted by category. Expect skipping around quite a bit; the topics are not sorted chronologically. We assume working knowledge of high school mathematics to the level of, optimistically, PreCalculus though one could get away with basic Algebra, Geometry, and Trigonometry. 

\part{Fundamentals: Algebra}

\chapter{Exponents \& Logarithms}

\section*{Exponents}

Exponentiation is shorthand for repeated multiplication. One could write $3 \cdot 3 \cdot 3$ as $3^3$. Analogously, instead of writing $x \cdot x \cdot x \cdot x \cdot x$, we write $x^5$. Without loss of generality, we call x the base and 5 the exponent or power.

\subsection*{Exponent Properties}

When our exponents have the same base, we have some properties regarding the multiplication and division of exponents: $$ x^a \cdot x^b = x^{(a+b)}$$ $$ \frac{x^a}{x^b} = x^{a-b}$$ $$ (x^a)^b = x^{ab}$$

We define some exponents:
$$ x^{-b} := \frac{1}{x^b} $$

$$ 1^x = 1$$

$$ x^1 = x $$

$$ x^0 \ \text{(where} \ x > 0) := 1$$

If $x$ is a negative number, we leave the exponent undefined as division by zero is not permitted. $0^0$, has no agreed-upon value. It is usually defined to be one or left undefined. In algebra and combinatorics, the generally agreed upon value is 1. In mathematical analysis, the expression is sometimes left undefined.

\section*{Roots}

Our exponent rules hold for fractional values. $$ 25^{\frac{1}{2}} = (5^2)^{\frac{1}{2}} = 5^1$$

The exponents $x^{\frac{1}{2}}$ and $x^{\frac{1}{3}}$ are called square roots and cube roots of x and are denoted $\sqrt{x}$ and $\sqrt[3]{x}$ respectively . Those with higher degrees are referred to the $n$th root of the number x and are denoted $\sqrt{n}{x}$. The $\sqrt{}$ symbol is called a radical. When asked for the $n$th root, one is being asked "what number to the $n$th power is equal to x". $ \sqrt{4}{10000} = 1000^{\frac{1}{4}} = \ \text{What number to the fourth power is 10000?} = \text{What is the fourth root of 10000} = 10$. Note that 10 is not the only number when raised to the fourth power is equal to 10000: $-4$ is an option. In general when asked for the root one expects the positive root.In an equation such as $x^2 = 9$ however, one expects both 3 and negative 3. This difficulty only occurs for even powers because negative numbers raised to an odd power are negative and positive numbers raised to an odd power are positive.

When working with fractional powers with numerators not equal to unity, we use our rule for exponential expressions backwards. $$ 8^{\frac{5}{3} = (8^{\frac{1}{3}})^5 = 2^5 = 32} $$

\subsection*{Simplifying Expressions Involving Radicals}

In a radical expression, all factors that can be removed should be removed. This is done by writing the prime factorization of the radicand (number under the radical). The root it then applied to each factor. $$ \sqrt{27} = 3^3 = 3 \sqrt{3}$$

It is often useful to rationalize the denominator of a fraction. Rationalizing is the process of making the denominators of fractions rational. To rationalize those involving square roots, one just needs to multiply the numerator and denominator of the fraction by the square roots. For the other roots, one first needs to reduce the radical via its prime factorization, split up, and then multiply each factor by its root separately. We must choose our multiplying factor such that the exponent under the radical is equal to that of the root. $$ \frac{2 \sqrt{5}}{3 \sqrt[4]{72}} = \frac{2 \sqrt{5}}{3 \sqrt[4]{2^3} \sqrt[4]{3^2}} $$ We first multiply by $ \sqrt[4]{2}$ and then $\sqrt[4]{3^2}$

If given an expression rather then a lone radical, one must multiply by the conjugate radical. $$\frac{1}{3 + \sqrt{5}} = \frac{1}{3 + \sqrt{5}} \frac{3 - \sqrt{5}}{3 - \sqrt{5}}$$

In a two term expression, the conjugate radical is found by changing the sign in front of one radical term. This method only works for two term expression and square roots. One must be creative when simplifying expression with multiple terms or those with roots higher than the second degree.

\section*{Logarithms}



\chapter{Proportions}

\chapter{Linear Equations}

\chapter{Quadratic Eqautions}

\chapter{Factoring \& Manipulating}
\chapter{Complex Numbers}
\chapter{Functions}
\chapter{Sets}
\chapter{Equations \& Expressions}

\chapter{Inequalities}
\chapter{Operations \& Relations}
\chapter{Sequences \& Series}
\chapter{Limits}
\chapter{Repeated Functions}
\chapter{Vectors \& Matrices}

\part{Fundamentals: Geometry \& Trigonometry}

\part{Fundamentals: Number Theory}

\part{Fundamentals: Combinatorics, Probability, \& Statistics}

\part{Advanced Topics: Algebra}

\part{Advanced Topics: Geometry}

\part{Advanced Topics: Number Theory}

\part{Advanced Topics: Combinatorics, Probability, \& Statistics}

\part{Advanced Topics: Collegiate}


\appendix

\begin{appendices}

\chapter{Proofs}

\chapter{Problems}

\end{appendices}

\backmatter

\bibliographystyle{plain}
\bibliography{refs}



\end{document}
